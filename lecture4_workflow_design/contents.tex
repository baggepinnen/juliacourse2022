% !TEX root=presentation_design_patterns.tex


\title{Antipatterns and idiomatic Julia versions}

\maketitle

%====================================================================
%====================================================================
\begin{frame}{Introduction}{}
    How to use multiple dispatch to structure your code

    Some common patterns and anti-patterns
    \pause
    \begin{block}{Code reuse}
        \begin{itemize}
            \item Julia is excellent on code reuse
            \begin{itemize}
                \item Interoperability between packges and types
                \item In your own code
            \end{itemize}
            \item Requires good and thoughtful design!
        \end{itemize}
    \end{block}

\end{frame}



\section{Dispatch on types}

\begin{frame}[fragile]{Dispatch on types}{Antipattern}

    \begin{minted}[fontsize=\small]{python}
class Shape:
    @property
    def name(self):
        return self.__class__

class Rectangle(Shape): pass
class Ellipse(Shape): pass

def intersect(s1, s2):
    if isinstance(s1, Rectangle) and isinstance(s2, Ellipse):
        print('Rectangle x Ellipse [names s1=%s, s2=%s]' % (s1.name, s2.name))
    elif isinstance(s1, Rectangle) and isinstance(s2, Rectangle):
        print('Rectangle x Rectangle [names s1=%s, s2=%s]' % (s1.name, s2.name))
    else:
        # Generic shape intersection.
        print('Shape x Shape [names s1=%s, s2=%s]' % (s1.name, s2.name))
    \end{minted}
    \bad{Drawback:} To implement a new geometry, the user has to
modify the source code pf \mintinline[]{text}{intersect}

\vspace{3mm}
Can only dispatch based on one argument
\end{frame}


\begin{frame}[fragile]{Dispatch on types}{Multiple dispatch approach}
    \begin{columns}
 	\begin{column}{0.8\textwidth}

\begin{minted}[]{julia1}
abstract type AbstractShape end
struct Rectangle <: AbstractShape end
struct Ellipse <: AbstractShape end

intersect(s1::Rectangle, s2::Ellipse) = do_something...
intersect(s1::Rectangle, s2::Rectangle) = do_something...
intersect(s1, s2) = do_something...
\end{minted}
\end{column}
\hspace{-5mm}
\begin{column}{0.3\textwidth}
    \nice{Benefit:} The user can send in an arbitrary existing subtype of
    \mintinline[]{text}{AbstractShape} as well as creating new such
    subtypes without modifying the function \mintinline[]{text}{intersect}.

    \vspace{3mm}
    Can dispatch based on both types.
\end{column}
\end{columns}


\end{frame}


\section{Dispatch on strings}\label{dispatch-on-strings}

\begin{frame}[fragile]{Dispatch on strings}{Antipattern}

\begin{minted}[]{julia1}
function process(data, preprocess="sumto1")
    if preprocess == "sumto1"
        data = data ./ sum(data)
    elseif preprocess == "norm1"
        data = data ./ norm(data)
    elseif preprocess == "filter"
        data = filter(data)
    end
    do_something_with(data)
end
\end{minted}

\bad{Drawback:} To implement a new form of preprocessing, the user has to
modify the source code pf \mintinline[]{text}{process}

\end{frame}

\begin{frame}[fragile]{Dispatch on strings}{Functional approach}

\begin{minted}[]{julia1}
function process(data, preprocess)
    data = preprocess(data)
    do_something_with(data)
end
\end{minted}

\nice{Benefit:} The user can send in an arbitrary function

\end{frame}

\begin{frame}[fragile]{Dispatch on strings}{Multiple dispatch approach}
    \begin{columns}
 	\begin{column}{0.7\textwidth}

\begin{minted}[]{julia1}
abstract type AbstractPreprocesser end

struct SumToOne <: AbstractPreprocesser end
preprocess(::SumToOne, data) = data ./ sum(data)

struct NormOne <: AbstractPreprocesser end
preprocess(::NormOne, data) = data ./ norm(data)

struct Filter <: AbstractPreprocesser
    cutoff
end
preprocess(f::Filter, data) = filter(data, f.cutoff)

function process(data, processor)
    data = preprocess(processor, data)
    do_something_with(data)
end
\end{minted}
\end{column}
\begin{column}{0.3\textwidth}
    \nice{Benefit:} The user can send in an arbitrary existing subtype of
    \mintinline[]{text}{AbstractPreprocesser} as well as creating new such
    subtypes without modifying the function \mintinline[]{text}{preprocess}.
\end{column}
\end{columns}


\end{frame}



\section{Several names for the same function}\label{several-names-for-the-same-function}

\begin{frame}[fragile]{Several names for the same function}{Antipattern}
    \begin{columns}
 	\begin{column}{0.65\textwidth}

\begin{minted}[fontsize=\footnotesize]{julia1}
function optimize_gd(problem, x)
    for i in iterations
        x = take_gd_step(x, problem)
        convergence_check_x(x) && break
        convergence_check_gradient(x, problem) && break
        convergence_check_funval(x) && break
    end
    x
end
function optimize_bfgs(problem, x)
    for i in iterations
        x = take_bfgs_step(x, problem)
        convergence_check_x(x) && break
        convergence_check_gradient(x, problem) && break
        convergence_check_funval(x) && break
    end
    x
end
function optimize_newton(problem, x)
    for i in iterations
        x = take_newton_step(x, problem)
        convergence_check_x(x) && break
        convergence_check_gradient(x, problem) && break
        convergence_check_funval(x) && break
    end
    x
end
\end{minted}

\end{column}
\begin{column}{0.4\textwidth}
\bad{Drawback:} The \emph{function} is the same, it find the optimum, what
differs is the \emph{method}.

\vspace{3mm}
The different names are only used for
dispatch. Lots of code is repeated, all algorithms have the same
structure.
\end{column}
\end{columns}

\end{frame}


\begin{frame}[fragile]{Several names for the same function}{Multiple dispatch approach}
        \begin{columns}
     	\begin{column}{0.8\textwidth}
            \begin{minted}[fontsize=\footnotesize]{julia1}
abstract type AbstractOptimizer end

struct GradientDescent <: AbstractOptimizer
    stepsize
end
struct BFGS <: AbstractOptimizer end
struct Newton <: AbstractOptimizer
    preconditioner
end

step(::GradientDescent, x, problem) = take_gd_step(x, problem)
step(::BFGS, x, problem) = take_bfgs_step(x, problem)
step(f::Newton, x, problem) = take_newton_step(x, problem)

function process(problem, x, algorithm)
    for i in iterations
        x = step(algorithm, x, problem)
        convergence_check_x(x) && break
        convergence_check_gradient(x, problem) && break
        convergence_check_funval(x) && break
    end
    x
end
            \end{minted}
\end{column}
\hspace{-7mm}
\begin{column}{0.4\textwidth}
\nice{Benefit:} The outer algorithm is only implemented once, the behaviour
that differs is the step, dispatched using types.

\vspace{3mm}
No need to copy the code in outer algorithm to implement a new optimization method.
\end{column}
\end{columns}

\end{frame}




%====================================================================
%====================================================================
\begin{frame}[fragile]{Other antipatterns}{}
    \begin{columns}
        \begin{column}{0.46\textwidth}
            \begin{block}{Antipatterns}
        \begin{minted}[breaklines,escapeinside=||,mathescape=true, numbersep=3pt, gobble=0, frame=lines, fontsize=\small, framesep=2mm]{julia}
repmat(v,1,10) .* A
for i ∈ collect(1:100)
randn(10,1)
0
Vector{typeof(x)}(undef, length(x))
mean(x.^2)
f(x::Float64) = x^2
middle(x::Vector) = x[end÷2]
        \end{minted}
        Functions that operate elementwise over arrays\\
        Rolling your own types
    \end{block}
    \end{column}

    \begin{column}{0.46\textwidth}
        \begin{block}{Nice patterns}
    \begin{minted}[breaklines,escapeinside=||,mathescape=true, numbersep=3pt, gobble=0, frame=lines, fontsize=\small, framesep=2mm]{julia}
v .* A
for i ∈ 1:100
randn(10)
zero(x)
similar(x)
mean(abs2, x)
f(x) = x^2
middle(x::AbstractVector) = x[end÷2]
    \end{minted}
    Broadcast a scalar function\\
    Reusing ecosystem types
\end{block}
\end{column}
\end{columns}

\end{frame}



%====================================================================
%====================================================================
\begin{frame}[fragile]{Common Julia idioms and types}{}
    \begin{minted}[breaklines,escapeinside=||,mathescape=true, numbersep=3pt, gobble=0, frame=lines, fontsize=\small, framesep=2mm]{julia}
condition && return x
    \end{minted}

Packages implementing commonly used types
\begin{itemize}
    \item Colors.jl
    \item Distances.jl
    \item RecipesBase.jl
    \item ChainRules.jl
\end{itemize}




\end{frame}
